\documentclass{full}
\hypersetup{pdfauthor=digitized by Andrew Voynov}
\newgeometry{margin=3cm,right=2cm}
\setunderlinedepth[2.2pt]
\setunderlinethickness[0.7pt]
\pagestyle{fancy}
\setcounter{page}{10}
\headerline
\cfoot{}
\rfoot{\textbf{\twelvetwelvesize\thepage}}
\renewcommand\normalsize{\setfontsize{13.5}{16}}
\begin{document}
\newcounter{task}
\setcounter{task}{1}
\newcommand\gettask{\arabic{task}}
\newcommand\nexttask{\setcounter{task}{\gettask+1}}
\noindent
\textbf{UNIT 2. OPERATING SYSTEMS}
\begin{enumerate}[label=\textbf{\Roman*.},leftmargin=5ex,start=\gettask]
  \setfontsize{13.5}{18}
  \item \textbf{Make sure that you know the following words:} \\
    computer hardware, service, system software, application programs,
    function, processor time, printing, input and output, application code,
    execute, device, cellular phone, supercomputer, web server, Microsoft
    Windows, Windows Phone, to access a computer, point of view

  \nexttask
  \item \textbf{Learn the following terms:}
    \begin{itemize}[itemsep=1ex,label=\setfontsize{20}{30}\textbullet\;]
      \setfontsize{13.5}{21}
      \item \uline{time-sharing operating system} - операционная система,
        работающая в режиме разделения времени
      \item \uline{accounting software} - бухгалтерское ПО
      \item \uline{memory allocation} - распределение памяти
      \item \uline{system call} - системный вызов
      \item \uline{real-time operating system} - операционная система реального
        времени
      \item \uline{to use specialized scheduling algorithm} - использовать
        специальный алгоритм планирования
      \item \uline{event-driven design} - событийно-управляемый проект
      \item \uline{clock interrupt} - прерывание по таймеру
      \item \uline{multiple-user access to computer} - многопользовательский
        доступ к компьютеру
      \item \uline{to run programs at the same time} - исполнять программы
        одновременно
      \item \uline{pre-emptive multitasking} - вытесняющая многозадачность
      \item \uline{cooperative multitasking} - кооперативная (совместная)
        многозадачность
      \item \uline{distributed operating system} - распределённая операционная
        система
      \item \uline{networked computer} - сетевой компьютер
      \item \uline{to carry out distributed computations} - выполнять
        распределённые вычисления
      \item \uline{cloud computing} - \quote{облачные} вычисления
      \item \uline{embedded operating system} - встроенная операционная система
    \end{itemize}

  \nexttask
  \item \textbf{Read and translate the text:}
\end{enumerate}

An \textbf{operating system} (\textbf{OS}) is software that manages computer
hardware resources and provides common services for computer programs. The
operating system is an essential component of the system software in a computer
system. Application programs usually require an operating system to function.

Time-sharing operating systems schedule tasks for efficient use of the system
and may also include accounting software for cost allocation of processor time,
mass storage, printing, and other resources.

For hardware functions such as input and output and memory allocation, the
operating system acts as an intermediary between programs and the computer
hardware, although the application code is usually executed directly by the
hardware and will frequently make a system call to an OS function or be
interrupted by it. Operating systems can be found on almost any device that
contains a computer - from cellular phones and video game consoles to
supercomputers and web servers. Examples of popular modern operating systems
include Android, BSD, iOS, Linux, OS X, QNX, Microsoft Windows, Windows Phone,
and IBM z/OS. All these, except Windows, Windows Phone and z/OS, share roots in
UNIX.

\section*{Types of operating systems}
A real-time operating system is a multitasking operating system that aims at
executing real-time applications. Real-time operating systems often use
specialized scheduling algorithms so that they can achieve a deterministic
nature of behavior. The main objective of real-time operating systems is their
quick and predictable response to events. They have an event-driven or
time-sharing design and often aspects of both. An event-driven system switches
between tasks based on their priorities or external events while time-sharing
operating systems switch tasks based on clock interrupts.

A multi-user operating system allows multiple users to access a computer system
at the same time. Time-sharing systems and Internet servers can be classified as
multi-user systems as they enable multiple-user access to a computer through the
sharing of time. Single-user operating systems have only one user but may allow
multiple programs to run at the same time.

A multi-tasking operating system allows more than one program to be running at
the same time, from the point of view of human time scales. A single-tasking
system has only one running program. Multi-tasking can be of two types:
pre-emptive and cooperative. In pre-emptive multitasking, the operating system
slices the CPU time and dedicates one slot to each of the programs. Unix-like
operating systems such as Solaris and Linux support pre-emptive multitasking, as
does AmigaOs. Cooperative multitasking is achieved by relying on each process to
give time to the other processes in a defined manner. 16-bit versions of
Microsoft Windows used cooperative multitasking. 32-bit versions of both Windows
NT and Win9x, used pre-emptive multitasking. Mac OS prior to OS X used to
support cooperative multitasking.

A distributed operating system manages a group of independent computers and
makes them appear to be a single computer. The development of networked
computers that could be linked and communicate with each other gave rise to
distributed computing. Distributed computations are carried out on more than one
machine. When computers in a group work in cooperation, they make a distributed
system.

In an o/s, distributed and cloud computing context, templating refers to
creating a single virtual machine image as a guest operating system, then saving
it as a tool for multiple running virtual machines. The technique is used both
in virtualization and cloud computing management, and is common in large server
warehouses.

Embedded operating systems are designed to be used in embedded computer systems.
They are designed to operate on small machines like PDAs with less autonomy.
They are able to operate with a limited number of resources. They are very
compact and extremely efficient by design. Windows CE and Minix 3 are Some
examples of embedded operating systems.

\vspace{1ex}
\nexttask
\begin{enumerate}[label=\textbf{\Roman*.},start=\gettask]
  \item \textbf{Answer the following questions:}
    \begin{enumerate}[label=\arabic*.,leftmargin=1ex,itemsep=-0.23ex]
      \item What is an operating system?
      \item What does it provide?
      \item What tasks do time-sharing systems schedule for efficient use of the
        system?
      \item When does an operating system act as an intermediary between
        programs and a computer?
      \item What kind of devices can an operating systems be found on?
      \item What are the examples of popular modern operating systems?
      \item What does a real-time operating system aim at?
      \item What is the main objective of real-time operating systems?
      \item How does a multi-user operating system allow multiple users to
        access a computer?
      \item What is the main feature of a single-user operating system?
      \item What are the main types of multi-tasking operating systems?
      \item What operating systems support pre-emptive multitasking?
      \item What operating system manages a group of independent computers and
        makes them appear to be a single computer?
      \item When do computers make a distributed system?
      \item What technique is used both in virtualization and cloud computing
        management?
      \item What type of machines are embedded operating systems designed to
        operate on?
    \end{enumerate}

  \nexttask
  \item \textbf{Retell the text briefly using the new words and expressions from
    ex. II.}

  \vspace{2ex}
  \nexttask
  \item \textbf{Fill in the gaps with the words given below. Use the dictionary
    if necessary.}
\end{enumerate}

\noindent\textit{\begin{enumerate*}[label=\alph*)]
  \item free software,
  \item be unproductive,
  \item programmer,
  \item the first version,
  \item UNIX,
  \item project,
  \item system software,
  \item computer science student,
  \item \quote{Linux},
  \item user space,
  \item BSD,
  \item ported
\end{enumerate*}}
\vspace{3ex}

The GNU Project was started by activist and ... (1) Richard Stallman with the
goal of creating a complete ... (2) replacement to the proprietary (3) operating
system. While the ... (4) was highly successful in duplicating the functionality
of various parts of UNIX, development of the GNU Hurd kernel proved to ... (5).
In 1991, Finnish computer science student Linus Torvalds, with cooperation from
volunteers collaborating over the Internet, released the ... (6) of the Linux
kernel. It was soon merged with the GNU ... (7) components and ... (8) to form a
complete operating system. Since then, the combination of the two major
components has usually been referred to as simply \quote{... (10)} by the
software industry. The Berkeley Software Distribution, known as ... (11), is the
UNIX derivative distributed by the University of California, Berkeley, starting
in the 1970s. Freely distributed and ... (12) to many minicomputers, it
eventually also gained a following for use on PCs, mainly as FreeBSD, NetBSD and
OpenBSD.
\vspace{1ex}

\setfontsize{11}{11}\noindent
(GNU is not Unix - операционная система GNU, Unix-совместимая ОС,
разрабатываемая FSF).
\normalsize

\vspace{0ex}
\section*{Grammar revision: The -ing form as a noun; after prepositions.}

\nexttask
\begin{enumerate}[label=\textbf{\Roman*.},start=\gettask]
  \item \textbf{Rewrite each of these sentences like this:}

    \textit{An important function of the operating system is to manage the
    computer's resources.}

    \textit{Managing the computer's resources is an important function of the
    operating system.}

    \begin{enumerate}[label=\arabic*.]
      \item The role of the operating system is to communicate directly with the
        hardware.
      \item One of the key functions of the operating system is to establish a
        user interface.
      \item An additional role is to provide services for applications software.
      \item Part of the work of mainframe operating system is to support
        multiple programs and users.
      \item The task in most cases is to facilitate interaction between a single
        user and a PC.
      \item One of the most important functions of a computer is to process
        large amounts of data quickly.
      \item The main reason for installing more memory is to allow the computer
        to process data faster.
    \end{enumerate}

  \vspace{3ex}
  \nexttask
  \item \textbf{Complete these sentences with the correct form of the verb:
    infinitive or -ing form.}
    \begin{enumerate}[label=\arabic*.,itemsep=-0.5ex]
      \item Don't switch off without (close down) your PC.
      \item I want to (upgrade) my computer.
      \item He can't get used to (log on) with a password.
      \item You can find information on the Internet by (use) a search engine.
      \item He objected to (pay) expensive telephone calls for Internet access.
      \item He tried to (hack into) the system without (known) the password.
      \item You needn't learn how to (program) in HTML before (design) webpages.
      \item I look forward to (input) data by voice instead of (use) a keyboard.
    \end{enumerate}
    \vspace{1ex}

  \nexttask
  \item \textbf{Look through the text. Make a short summary of it:}
  \vspace{-2ex}
\end{enumerate}

\section*{UNIX and UNIX-like operating systems}
Unix was originally written in assembly language.Ken Thompson wrote B, mainly
based on BCPL, based on his experience in the MULTICS project. B was replaced by
C, and Unix, rewritten in C, developed into a large, complex family of
interrelated operating systems which have been influential in every modern
operating
system.

The UNIX-like family is a diverse group of operating systems, with several major
sub-categories including System V, BSD, and Linux. The name \quote{UNIX} is a
trademark of The Open Group which licenses it for use with any operating system
that has been shown to conform to their definitions. \quote{UNIX-like} is
commonly used to refer to the large set of operating systems which resemble the
original UNIX.

Unix-like systems run on a wide variety of computer architecture. They are used
heavily for servers in business, as well as workstations in academic and
engineering environments. Free UNIX variants, such as Linux and BSD, are popular
in these areas.

Four operating systems are certified by The Open Group (holder of the Unix
trademark) as Unix. HP's HP-UX and IBM's AIX are both descendants of the
original System V Unix and are designed to run only on their respective vendor's
hardware. In contrast, Sun Microsystems's Solaris Operating System can run on
multiple types of hardware, including x86 and Sparc servers, and PCs. Apple's OS
X, a replacement for Apple's earlier (non-Unix) Mac OS, is a hybrid kernel-based
BSD variant derived from NEXTSTEP, Mach, and FreeBSD.

A subgroup of the Unix family is the Berkeley Software Distribution family,
which includes FreeBSD, NetBSD, and OpenBSD. These operating systems are most
commonly found on webservers, although they can also function as a personal
computer OS. The Internet owes much of its existence to BSD, as many of the
protocols now commonly used by computers to connect, send and receive data over
a network were widely implemented and refined in BSD. The World Wide Web was
also first demonstrated on a number of computers running an OS based on BSD
called NextStep.

BSD has its roots in Unix. In 1974, University of California, Berkeley installed
its first Unix system. Over time, students and staff in the computer science
department there began adding new programs to make things easier, such as text
editors. When Berkeley received new VAX computers in 1978 with Unix installed,
the school's undergraduates modified Unix even more in order to take advantage
of the computer's hardware possibilities. The Defense Advanced Research Projects
Agency of the US Department of Defense took interest, and decided to fund the
project. Many schools, corporations, and government organizations took notice
and started to use Berkeley's version of Unix instead of the official one
distributed by AT\&T.

\vspace{0.42ex}
\nexttask
\begin{enumerate}[label=\textbf{\Roman*.},start=\gettask]
  \item \textbf{Translate the text in written form:}
  \vspace{-0.7ex}
\end{enumerate}

\textbf{Windows 10} is a personal computer operating system developed and
released by Microsoft as part of the Windows NT family of operating systems. It
was officially unveiled in September 2014 following a brief demo at Build 2014.
The first version of the operating system entered a public beta testing process
in October 2014, leading up to its consumer release on July 29, 2015.

Windows 10 introduces what Microsoft described as \quote{universal apps};
expanding on Metro-style apps, these apps can be designed to run across multiple
Microsoft product families with nearly identical code - including PCs, tablets,
smartphones, embedded systems, Xbox One, Surface Hub and Holographic. The
Windows user interface was revised to handle transitions between a
mouse-oriented interface and a touchscreen-optimized interface based on
available input devices - particularly on 2-in-1 PCs; both interfaces include an
updated Start menu which incorporates elements of Windows 7's traditional Start
menu with the tiles of Windows 8. The first release of Windows 10 also
introduces a virtual desktop system, a window and desktop management feature
called Task View, the Microsoft Edge web browser, support for fingerprint and
face recognition login, new security features for enterprise environments, and
DirectX 12 and WDDM 2.0 to improve the operating system's graphics capabilities
for games.

Microsoft described Windows 10 as an \quote{operating system as a service} that
would receive ongoing updates to its features and functionality, augmented with
the ability for enterprise environments to receive non-critical updates at a
slower pace, or use long-term support milestones that will only receive critical
updates, such as security patches, over their five-year lifespan of mainstream
support. Terry Myerson, executive vice president of Microsoft's Windows and
Devices Group, argued that the goal of this model was to reduce fragmentation
across the Windows platform, as Microsoft aimed to have Windows 10 installed on
at least one billion devices in the two to three years following its release.

As of June 2016, Windows 10 use is on the rise, with previous versions of
Windows declining in their share of total usage as measured by web traffic. The
operating system is running on 350 million active devices and has an estimated
usage share of 22\% on personal computers and 12\% across all platforms (PC,
mobile, tablet, and console).
\end{document}
