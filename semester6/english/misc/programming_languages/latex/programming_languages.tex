\documentclass[a4paper, 12pt]{extarticle}
\usepackage{hyperref, fontspec, dirtytalk}
\usepackage[none]{hyphenat}
\usepackage[inline]{enumitem}
\usepackage[margin=2cm, left=3cm]{geometry}
\setmainfont{Liberation Serif}
\hypersetup{pdfauthor=digitized by Andrew Voynov}
\sloppy
\pagestyle{empty}
\begin{document}

\begin{center}
  \textbf{Programming Languages}
\end{center}

A programming language is a machine-readable artificial language designed to
express computations that can be performed by a machine, particularly a
computer.

Many programming languages have some form of written specification of their
syntax and semantics, since computers require precisely defined instructions.
Some are defined by a specification document (for example, an ISO Standard),
while others have a dominant implementation (such as Perl).

\textbf{Definitions}. Traits often considered important for constituting a
programming language:

Function: A programming language is a language used to write computer programs,
which involve a computer performing some kind of computation or algorithm and
possibly control external devices such as printers, robots, and so on.

Constructs: Programming languages may contain constructs for defining and
manipulating data structures or controlling the flow of execution.

Expressive power: The theory of computation classifies languages by the
computations they are capable of expressing. All Turing complete languages can
implement the same set of algorithms. ANSI/ISO SQL and Charity are examples of
languages that are not Turing complete, yet often called programming languages.

Some authors restrict the term \say{programming language} to those languages
that can express all possible algorithms; sometimes the term \say{computer
language} is used for more limited artificial languages.

Non-computational languages, such as markup languages like HTML or formal
grammars like BNF, are usually not considered programming languages. A
programming language (which may or may not be Turing complete) may be embedded
in these non-computational (host) languages.

\textbf{Usage}. A programming language provides a structured mechanism for
defining pieces of data, and the operations or transformations that may be
carried out automatically on that data. Programs for a computer might be
executed in a batch process without human interaction, or a user might type
commands in an interactive session of an interpreter. In this case the
\say{commands} are simply programs, whose execution is chained together. When a
language is used to give commands to a software application (such as a shell) it
is called a scripting language.

Programs must balance speed, size, and simplicity on systems ranging from
microcontrollers to supercomputers.

\end{document}
