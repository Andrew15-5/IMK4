\documentclass[a4paper, 12pt]{extarticle}
\usepackage{hyperref, fontspec, dirtytalk}
\usepackage[none]{hyphenat}
\usepackage[margin=2cm]{geometry}
\setmainfont{Liberation Serif}
\hypersetup{pdfauthor=digitized by Andrew Voynov}
\newcommand\indenton{\setlength\parindent{1.25cm}} \sloppy \pagestyle{empty}
\begin{document}

\begin{center}
  \textbf{Operating system}
\end{center}

An operating system (OS) is software, consisting of programs and data, that runs
on computers and manages computer hardware resources and provides common
services for efficient execution of various application software. Management
tasks include scheduling resource use to avoid conflicts and interference
between programs. Unlike most programs, which complete a task and terminate, an
operating system runs indefinitely and terminates only when the computer is
turned off.

For hardware functions such as input and output and memory allocation, the
operating system acts as an intermediary between application programs and the
computer hardware, although the application code is usually executed directly by
the hardware, but will frequently call the OS or be interrupted by it.

\textbf{Development of operating systems.}

In early computers, the user typed programs onto punched tape or cards, from
which they were read into the computer. The computer subsequently assembled or
compiled the programs and then executed them, and the results were then
transmitted to a printer. It soon became evident that much valuable computer
time was wasted between users and also while jobs (programs to be executed) were
being read or while the results were being printed. The earliest operating
systems consisted of software residing in the computer that handled
\say{batches} of user jobs --- i.e., sequences of jobs stored on magnetic tape
that are read into computer memory and executed one at a time without
intervention by user or operator. Accompanying each job in a batch were
instructions to the operating system (OS) detailing the resources needed by the
job --- for example, the amount of CPU time, the files and the storage devices
on which they resided, the output device, whether the job consisted of a program
that needed to be compiled before execution, and so forth. From these beginnings
came the key concept of an operating system as a resource allocator. This role
became more important with the rise of multiprogramming, in which several jobs
reside in the computer simultaneously and share resources --- for example, being
allocated fixed amounts of CPU time in turn. More sophisticated hardware allowed
one job to be reading data while another wrote to a printer and still another
performed computations. The operating system was the software that managed these
tasks in such a way that all the jobs were completed without interfering with
one another.

Further work was required of the operating system with the advent of interactive
computing, in which the user enters commands directly at a terminal and waits
for the system to respond. Processes known as terminal handlers were added to
the system, along with mechanisms like interrupts (to get the attention of the
operating system to handle urgent tasks) and buffers (for temporary storage of
data during input/output to make the transfer run more smoothly). A large
computer can now interact with hundreds of users simultaneously, giving each the
perception of being the sole user. The first personal computers used relatively
simple operating systems, such as some variant of DOS (disk operating system),
with the main jobs of managing the user's files, providing access to other
software (such as word processors), and supporting keyboard input and screen
display. Perhaps the most important trend in operating systems today is that
they are becoming increasingly machine-independent. Hence, users of modern,
portable operating systems like UNIX, Microsoft Corporation's Windows NT, and
Linux are not compelled to learn a new operating system each time they purchase
a new, faster computer (possibly using a completely different processor).

\end{document}
