\documentclass[a4paper]{article}
\usepackage{hyperref, fontspec, fontsize, dirtytalk}
\usepackage[none]{hyphenat}
\usepackage[inline]{enumitem}
\usepackage[margin=2cm, left=3cm]{geometry}
\changefontsize{13pt}
\setmainfont{Liberation Serif}
\hypersetup{pdfauthor=digitized by Andrew Voynov}
\def\ruleftquote{«}
\def\rurightquote{»}
\newcommand\ruquote[1]{\ruleftquote #1\rurightquote}
\newcommand\indenton{\setlength\parindent{1.25cm}}
\sloppy
\begin{document}

\noindent\textbf{UNIT 3. NETWORKS}

\vspace{2ex}

\def\bfitem#1{\item \textbf{#1}}

\begin{enumerate}[label={\textbf{\Roman*.}},leftmargin=0pt,itemindent=*]
  \indenton
  \bfitem{Make sure that you know the following words:}

    Direct connection, include hosts, personal computers, differ in, to carry
    signals, transmission medium, communication protocols, digital audio,
    printers, fax machines, use of email, to the destination, instant messaging,
    chat rooms, video telephone calls, video conferencing, service attack,
    control information.

  \vspace{2ex}

  \bfitem{Learn the following terms:}

  \begin{itemize}
    \item \textbf{to exchange data} --- обмениваться информацией
    \item \textbf{data link} --- канал передачи данных
    \item \textbf{wireless medium} --- беспроводной носитель данных
    \item \textbf{network node} --- главный узел

    \item \textbf{shared use of applications} --- совместное использование
      приложений

    \item \textbf{to organize network traffic} --- организовывать сетевой трафик

    \item \textbf{application-specific communications protocol} ---
      специализированный протокол связи

    \item \textbf{point-to-point telecommunication link} --- канал связи
      \ruquote{точка-точка}

    \item \textbf{packet switched network} --- сеть с пакетной коммутацией

    \item \textbf{bandwidth of the transmission medium} --- полоса пропускания
      передающей среды

    \item \textbf{network address} --- сетевой адрес
    \item \textbf{error detection code} --- код с обнаружением ошибок

    \item \textbf{to deliver user data} --- оставлять сообщения, содержащие
      информацию для пользователя

    \item \textbf{packet header} --- заголовок пакета

    \item \textbf{trailer} --- трейлер --- запись с контрольной суммой в конце
      массива данных

    \item \textbf{shared storage device} --- совместно используемое запоминающее
      устройство

    \item \textbf{to accomplish a task} --- выполнить задание

    \item \textbf{computer cracker} --- взломщик компьютерных систем,
      злоумышленник

  \end{itemize}

  \vspace{2ex}

  \bfitem{Read and translate the text:}

    A computer network or data network is a telecommunications network which
    allows computers to exchange data. In computer networks, networked computing
    devices exchange data with each other using a data link. The connections
    between nodes are established using either cable media or wireless media.
    The best-known computer network is the Internet.

    Network computer devices that originate, route and terminate the data are
    called network nodes. Nodes can include hosts such as personal computers,
    phones, servers as well as networking hardware. Two such devices can be said
    to be networked together when one device is able to exchange information
    with the other device, whether or not they have a direct connection to each
    other.

    Computer networks differ in the transmission medium used to carry their
    signals, the communications protocols to organize network traffic, the
    network's size, topology and organizational intent.

    Computer networks support an enormous number of applications such as access
    to the World Wide Web, video, digital audio, shared use of application and
    storage servers, printers, and fax machines, and use of email and instant
    messaging applications as well as many others. In most cases,
    application-specific communications protocols are layered (i.e. carried as
    payload) over other more general communications protocols.

    Computer communication links that do not support packets, such as
    traditional point-to-point telecommunication links, simply transmit data as
    a bit stream. However, most information in computer networks is carried in
    packets. A network packet is a formatted unit of data (a list of bits or
    bytes, usually a few tens of bytes to a few kilobytes long) carried by a
    packet-switched network.

    In packet networks, the data is formatted into packets that are sent through
    the network to their destination. Once the packets arrive they are
    reassembled into their original message. With packets, the bandwidth of the
    transmission medium can be better shared among users than if the network
    were circuit switched. When one user is not sending packets, the link can be
    filled with packets from others users, and so the cost can be shared, with
    relatively little interference, provided the link isn't overused.

    Packets consist of two kinds of data: control information, and user data
    (payload). The control information provides data the network needs to
    deliver the user data, for example: source and destination network
    addresses, error detection codes, and sequencing information. Typically,
    control information is found in packet headers and trailers, with payload
    data in between.

    Often the route a packet needs to take through a network is not immediately
    available. In that case the packet is queued and waits until a link is free.

    A computer network facilitates interpersonal communications allowing users
    to communicate efficiently and easily via various means: email, instant
    messaging, chat rooms, telephone, video telephone calls, and video
    conferencing. Providing access to information on shared storage devices is
    an important feature of many networks. A network allows sharing of files,
    data, and other types of information giving authorized users the ability to
    access information stored on other computers on the network. A network
    allows sharing of network and computing resources. Users may access and use
    resources provided by devices on the network, such as printing a document on
    a shared network printer. Distributed computing uses computing resources
    across a network to accomplish tasks. A computer network may be used by
    computer crackers to deploy computer viruses or computer worms on devices
    connected to the network, or to prevent these devices from accessing the
    network via a denial of service attack.

  \vspace{4ex}

  \bfitem{Answer the following questions:}

    \begin{enumerate}[label=\arabic*.]
      \item What does computer network allow computers to do?
      \item What devices exchange data with each other using a data link?
      \item How are connections between nodes established?
      \item What computer devices are called network nodes?
      \item What devices do nodes include?
      \item How do computer networks differ in?
      \item What applications do computer networks support?
      \item How is most information in computer networks carried?
      \item What is a network packet?
      \item What kinds of data do packets consist of?

      \item How does computer network allow users to communicate via various
        means?

      \item What is an important feature of many networks?

      \item Who may a computer network be used by to deploy computer viruses or
        computer worms?

    \end{enumerate}

  \bfitem{Retell the text briefly using the new words and expressions from ex.
  II.}

  \bfitem{Fill in the gaps with the words given below. Use the dictionary if
  necessary.}

    \begin{center}
      \textbf{Future Initiatives}
    \end{center}

    {\itshape\noindent\begin{enumerate*}[label=\alph*)]
      \item media,
      \item to evolve,
      \item to outsource,
      \item applications,
      \item YouTube, \\

      \item efficiently,
      \item \say{the cloud},
      \item consumption,
      \item functions,
      \item powerful
    \end{enumerate*}}

    \vspace{2ex}

    Network software ... (1) including Twitter and ... (2) have been rapidly
    replacing traditional ... (3) as a source for news, information, training
    and entertainment. As the Internet continues to grow as the primary network
    platform, the role of network software will continue ... (4). A trend toward
    \say{cloud-based computing} seeks to ... (5) network and application
    management from the corporate premises to data centers. Traditional network
    software ... (6) including application hosting, communication,
    administration and backups will be outsourced to ... (7) where they can be
    performed more ... (8). Network software platforms are becoming more ... (9)
    targeting new objectives to reduce the number of servers, cooling
    requirements and power ... (10) requirements of corporations and data
    centers. This concept of \say{Green I/T} is a new and evolving purpose of
    network software technologies.

  \newpage

  \textbf{Grammar revision: Relative clauses with a participle}

  \bfitem{Rewrite each of these sentences like this:}

    \begin{enumerate}[label=\arabic*.,itemsep=0pt]

      \item A gateway is an interface (enable) dissimilar networks to
        communicate.

      \item A bridge is a hardware and software combination (use) to connect the
        same type of networks.

      \item A backbone is a network transmission path (handle) major data
        traffic.

      \item A router is a special computer (direct) messages when several
        networks are linked.

      \item A network is a number of computers and peripherals (link) together.

      \item A LAN is a network (connect) computers over a small distance such as
        within a company.

      \item A server is a powerful computer (store) many programs (share) by all
        the clients in the network.

      \item A client is a network computer (use) for accessing a service on a
        server.

      \item A thin client is a simple computer (comprise) a processor and
        memory, display, keyboard, mouse and hard drives only.

      \item A hub is an electronic device (connect) all the data cabling in a
        network.

    \end{enumerate}

  \bfitem{Link these statements using a relative clause with a participle.}

    \vspace{-0.5ex}

    \begin{enumerate}[label=\arabic*.,itemsep=0pt]

      \item \begin{enumerate}[label=\alph*)]
        \item The technology is here today.
        \item It is needed to set up a home network.
      \end{enumerate}

      \item \begin{enumerate}[label=\alph*)]
        \item You only need one network printer.
        \item It is connected to the server.
      \end{enumerate}

      \item \begin{enumerate}[label=\alph*)]
        \item Her house has a network.
        \item It allows basic file-sharing and multi-player gaming.
      \end{enumerate}

      \item \begin{enumerate}[label=\alph*)]
        \item There is a line receiver in the living room.
        \item It delivers home entertainment audio to speakers.
      \end{enumerate}

      \item \begin{enumerate}[label=\alph*)]
        \item Eve has designed a site.
        \item It is dedicated to dance.
      \end{enumerate}

      \item \begin{enumerate}[label=\alph*)]
        \item She has built-in links.
        \item They connect her site to other dance sites.
      \end{enumerate}

      \item \begin{enumerate}[label=\alph*)]
        \item She created the site using a program called Netscape Composer.
        \item It is contained in Netscape Communicator.
      \end{enumerate}

      \item \begin{enumerate}[label=\alph*)]
        \item At the center of France Telecom's home of tomorrow is a network.
        \item It is accessed through a Palm Pilot-style control pad.
      \end{enumerate}

      \item \begin{enumerate}[label=\alph*)]
        \item The network can simulate the owner's presence.
        \item This makes sure vital tasks are carried out in her absence.
      \end{enumerate}

      \item \begin{enumerate}[label=\alph*)]
        \item The house has an electronic door-keeper.
        \item It is programmed to recognize you.
        \item This gives access to family only.
      \end{enumerate}

    \end{enumerate}

  \newpage

  \bfitem{Look through the text. Make a short summary of it:}

  \vspace{6ex}

    \begin{center}
      \textbf{Communications protocols}
    \end{center}

    A communications protocol is a set of rules for exchanging information over
    network links. In a protocol stack (also see the OSI model), each protocol
    leverages the services of the protocol below it. An important example of a
    protocol stack is HTTP (the World Wide Web protocol) running over TCP over
    IP (the Internet protocols) over IEEE 802.11 (the Wi-Fi protocol). This
    stack is used between the wireless router and the home user's personal
    computer when the user is surfing the web.

    Whilst the use of protocol layering is today ubiquitous across the field of
    computer networking, it has been historically criticized by many researchers
    for two principal reasons. Firstly, abstracting the protocol stack in this
    way may cause a higher layer to duplicate functionality of a lower layer, a
    prime example being error recovery on both a per-link basis and an
    end-to-end basis. Secondly, it is common that a protocol implementation at
    one layer may require data, state or addressing information that is only
    present at another layer, thus defeating the point of separating the layers
    in the first place. For example, TCP uses the ECN field in the IPv4 header
    as an indication of congestion; IP is a network layer protocol whereas TCP
    is a transport layer protocol. Communication protocols have various
    characteristics. They may be connection-oriented or connectionless, they may
    use circuit mode or packet switching, and they may use hierarchical
    addressing or flat addressing.

    There are many communication protocols, a few of which are described below.

    The complete IEEE 802 protocol suite provides a diverse set of networking
    capabilities. The protocols have a flat addressing scheme. They operate
    mostly at levels 1 and 2 of the OSI model.

    For example, MAC bridging (IEEE 802.1D) deals with the routing of Ethernet
    packets using a Spanning Tree Protocol. IEEE 802.1Q describes VLANs, and
    IEEE 802.1X defines a port-based Network Access Control protocol, which
    forms the basis for the authentication mechanisms used in VLANs (but it is
    also found in WLANs) --- it is what the home user sees when the user has to
    enter a \say{wireless access key}.

    Ethernet, sometimes simply called LAN, is a family of protocols used in
    wired LANs, described by a set of standards together called IEEE 802.3
    published by the Institute of Electrical and Electronics Engineers.

    Wireless LAN, also widely known as WLAN or Wi-Fi, is probably the most
    well-known member of the IEEE 802 protocol family for home users today. It
    is standardized by IEEE 802.11 and shares many properties with wired
    Ethernet. The Internet Protocol Suite, also called TCP/IP, is the foundation
    of all modern networking. It offers connection-less as well as
    connection-oriented services over an inherently unreliable network traversed
    by data-gram transmission at the Internet protocol (IP) level. At its core,
    the protocol suite defines the addressing, identification, and routing
    specifications for Internet Protocol Version 4 (IPv4) and for IPv6, the next
    generation of the protocol with a much enlarged addressing capability.

    Synchronous optical networking (SONET) and Synchronous Digital Hierarchy
    (SDH) are standardized multiplexing protocols that transfer multiple digital
    bit streams over optical fiber using lasers. They were originally designed
    to transport circuit mode communications from a variety of different
    sources, primarily to support real-time, uncompressed, circuit-switched voice
    encoded in PCM (Pulse-Code Modulation) format. However, due to its protocol
    neutrality and transport-oriented features, SONET/SDH also was the obvious
    choice for transporting Asynchronous Transfer Mode (ATM) frames.

    Asynchronous Transfer Mode (ATM) is a switching technique for
    telecommunication networks. It uses asynchronous time-division multiplexing
    and encodes data into small, fixed-sized cells. This differs from other
    protocols such as the Internet Protocol Suite or Ethernet that use variable
    sized packets or frames. ATM has similarity with both circuit and packet
    switched networking. This makes it a good choice for a network that must
    handle both traditional high-throughput data traffic, and real-time,
    low-latency content such as voice and video. ATM uses a connection-oriented
    model in which a virtual circuit must be established between two endpoints
    before the actual data exchange begins.

    While the role of ATM is diminishing in favor of next-generation networks,
    it still plays a role in the last mile, which is the connection between an
    Internet service provider and the home user.

  \newpage

  \bfitem{Translate the text in written form:}

    \begin{center}
      \textbf{Views of networks}
    \end{center}

    Users and network administrators typically have different views of their
    networks. Users can share printers and some servers from a workgroup, which
    usually means they are in the same geographic location and are on the same
    LAN, whereas a Network Administrator is responsible to keep that network up
    and running. A community of interest has less of a connection of being in a
    local area, and should be thought of as a set of arbitrarily located users
    who share a set of servers, and possibly also communicate via peer-to-peer
    technologies.

    Network administrators can see networks from both physical and logical
    perspectives. The physical perspective involves geographic locations,
    physical cabling, and the network elements (e.g., routers, bridges and
    application layer gateways) that interconnect via the transmission media.
    Logical networks, called, in the TCP/IP architecture, subnets, map onto one
    or more transmission media. For example, a common practice in a campus of
    buildings is to make a set of LAN cables in each building appear to be a
    common subnet, using virtual LAN (VLAN) technology.

    Both users and administrators are aware, to varying extents, of the trust
    and scope characteristics of a network. Again using TCP/IP architectural
    terminology, an intranet is a community of interest under private
    administration usually by an enterprise, and is only accessible by
    authorized users (e.g. employees). Intranets do not have to be connected to
    the Internet, but generally have a limited connection. An extranet is an
    extension of an intranet that allows secure communications to users outside
    of the intranet (e.g. business partners, customers).

    Unofficially, the Internet is the set of users, enterprises, and content
    providers that are interconnected by Internet Service Providers (ISP). From
    an engineering viewpoint, the Internet is the set of subnets, and aggregates
    of subnets, which share the registered IP address space and exchange
    information about the reachability of those IP addresses using the Border
    Gateway Protocol. Typically, the human-readable names of servers are
    translated to IP addresses, transparently to users, via the directory
    function of the Domain Name System (DNS).

    Over the Internet, there can be business-to-business (B2B),
    business-to-consumer (B2C) and consumer-to-consumer (C2C) communications.
    When money or sensitive information is exchanged, the communications are apt
    to be protected by some form of communications security mechanism. Intranets
    and extranets can be securely superimposed onto the Internet, without any
    access by general Internet users and administrators, using secure Virtual
    Private Network (VPN) technology.

\end{enumerate}

\end{document}
