\documentclass[a4paper]{article}
\usepackage{hyperref, fontspec, csquotes}
\usepackage[british]{babel}
\usepackage[none]{hyphenat}
\usepackage[fontsize=14pt]{fontsize}
\usepackage[margin=1cm]{geometry}
\setmainfont{Liberation Serif}
\hypersetup{pdfauthor=digitized by Andrew Voynov}
\sloppy
\pagestyle{empty}
\newcommand\Math[1]{$\mathrm{#1}$}

\setlength\parindent{1.25cm}

\setlength\parskip{4pt}
\begin{document}

\begin{center}
  Nanotechnology in school
\end{center}

\begin{flushright}
  \textbf{Matthias Mallmann}

  www.esrf.eu/UsersAndScience/Publications/Newsletter
\end{flushright}

\textbf{Matthias Mallmann} from NanoBioNet eV explains what nanotechnology
really is, and offers two nano-experiments for the classroom.

Nanotechnology has become a popular buzzword in science and politics. This key
technology is considered not only a major source of innovation in technology,
medicine and other fields, but also one of the main challenges for the 21st
century. European universities and high-level vocational training programmes
already cover this technology extensively. However, although the word
nanotechnology will be familiar to many high-school students, the subject is not
widely taught in European schools. This article outlines several initiatives to
increase awareness of nanotechnology among European science teachers, and
details two nanotechnology experiments for the classroom.

What is nanotechnology?

Nanotechnology is not really anything new. It deals with entities and processes
on the scale of \Math{10^{-9}} m (1 nanometre), which is the dimension of
molecules and atoms --- a scale that chemists, biochemists and cell biologists
have worked with for centuries.

At the nanoscale, the properties of a material may change. For example,
hardness, electrical conductivity, colour or chemical reactivity of minuscule
particles of materials are related to the diameter of the particle. Specific
functionalities, therefore, can be achieved by reducing the size of the
particles to 1--100 nm.

A well-known application of early nanotechnology is the ruby red colour that was
used for stained glass windows during the Middle Ages (see image). The colour is
a result of gold atoms clustering to form nanoparticles instead of the more
usual solid form. These small gold particles allow the long-wave red light to
pass through but block the shorter wavelengths of blue and yellow light. The
colour, therefore, depends both on the element involved (gold) and on the
particle size; silver nanoparticles, for example, can give a yellow colour.

What is new, though, is the multidisciplinary approach and the ability to
\enquote{look} at these entities. The atomic force microscope, which was
developed in the late 1980s, allows scientists to view structures at a
nano-metric scale and to handle even single atoms via scanning probe microscopy.
Now biologists can discuss steric effects of cell membranes with chemists, while
physicists provide the tools to watch the interaction \textit{in vivo}.
Nanoparticles play an important role in the pharmaceutical industry (delivering
active agents to the required part of the body) in the production of emulsion
paint and cosmetics and in the optimisation of catalysts. Nanotechnology,
therefore, has combined all natural sciences and creates cross-links between the
different disciplines.

\end{document}
